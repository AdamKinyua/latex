\documentclass{article}
\usepackage[utf8]{inputenc}
\usepackage{amsmath}

\title{Mathematical notation}
\author{adam kinyua}
\date{today}

\begin{document}

\maketitle

\section{Introduction}


\iffalse
------------------------------------------------------
-> typing equations inside the paragraph
------------------------------------------------------
\fi


Typing an equation inside a paragraph is supposed to be like \(x+y=z\). It makes everything look amazing. Alternatively, you can also write it like $x+y=z$. This too works the same as the other one. We can also create a space above and below the equation by $$x+y=z$$It ensures that the equation stands out by itself. Alternatively, this too \[x+y=z\]does as the preceding.


\iffalse
------------------------------------------------------
-> the equations and align environments - allowing labelling and numbering of equations so we can refer to them later on. you must use 
-> this is where we make use of \usepackage{amsmath}
------------------------------------------------------
\fi


\begin{equation}
x+y=z \label{eq:first}
\end{equation}
\noindent According to equation \ref{eq:first} above. However, this is considered a separate paragraph
If we want our equation referenced to to have a parenthesis then we use \eqref{eq:first}


\iffalse
------------------------------------------------------
-> if we have more than one equation; check the alignments below:
------------------------------------------------------
\fi


\begin{align}
x+y=z \\
a+b+d=c 
\end{align}

\begin{align}
x&+y+v=z \\
a&+b+=c 
\end{align}

\iffalse
------------------------------------------------------
-> more environments - we only want numbers in equations we want to refer to later
------------------------------------------------------
\fi

If we do not want to have a number in an equation:

\begin{align}
x&+y+v=z \nonumber \\
a&+b+=c 
\end{align}

If we want these equations to have no numbers:
\begin{align*}
x&+y+v=z \nonumber \\
a&+b+=c 
\end{align*}


\iffalse
------------------------------------------------------
MATH SYMBOLS AND GREEK LETTERS
- start by creating a mathematical environment using \( and \)
------------------------------------------------------
\fi

\(
\alpha \beta \gamma \delta \omega \hspace{1cm}  \Alpha \Beta \Gamma \Delta \Omega
\)


\end{document}
